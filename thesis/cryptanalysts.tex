\subsection{Characterization of Cryptanalysts}

We divide cryptanalysts into four subclasses given as a variation of
the subclasses of cryptographic secrecy.
We make our division in reverse order in terms of hardness as
follows:

\begin{itemize}
\item{Practical Secrecy:} In this subclass of cryptanalysts we have those
which most frequently occur. These cryptanalysts are in possesion of
finite computational resources but may use social, sidechannel, or
other methods to obtain a break on a system which does not break the
mathematics of the scheme itself, may not in the future be reproduced,
but are sucessful in their attempt none the less.

\item{Computational Secrecy:} This subclass of cryptanalysts are those who
we can assume have at their disposal more resources and are capable of
more large scale attacks than those in the previous subclass. They are
still in possession of all the methods of the practical cryptanalysts
but we make the assumption that they are not working alone but as a
well organized unit such as those of a large organization. These
attacks are becoming more and more frequent the organization itself may
be known to the defense but for various reasons ( including but not
limited to political) a direct confrontation must be avoided.

\item{Unconditional Secrecy:} In this subclass we find those cryptanalysts
who directly attack the mathematics of a given cryptosystem.
Specifically we have the cryptanalysts who play against most modern
systems based on the discrete logarithm problem, the integer
factorization problem, and other number theoretic schemes. We limit
this group to problems which are played over systems whose proofs of
security are given over schemes an average-case hardness.

\item{Perfect Secrecy:} This is the subclass of cryptanalysts we will be
concerned with for the remainder of this paper. These players are
concerned with breaking the mathematics of schemes based on intractible
problems with a provable security based on worst-case hardness
assumptions. Specifically, we concern ourselves with the concept of
quantum-safe cryptographic schemes over lattices.
\end{itemize}