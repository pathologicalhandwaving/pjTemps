\section{The Soliloquy Problem}

In 2010 at the International Public Key Cryptography conference Smart
and Veracauteren introduced a cryptographic scheme [\cite{sma20101}] based on
principal ideal lattices with relatively small key sizes and promising
to enable FHE.
\smallskip
In late 2014 GCHQ announced that they had developed an equivalent
cryptosystem to [\cite{sma20101}] in 2007 named Soliloquy[\cite{cam20140}], but from 2010 to
2013 they had constructed a successful attack against the system and
decided to cease development.
\smallskip
Furthermore, they claimed (without proof) that the assumed hard
problem of finding the short generator of a principal ideal lattice was
both easy and efficient to solve. This conflicted with the [\cite{sma20101}]
claim that the problem of finding a short generator was hard. In
addition the solvability claim GCHQ was not only given for quantum
computers but also for ordinary modern computer system with a practical
amounts of resources.
\smallskip
The announcement by GCHQ created quite a stir in the cryptographic
community. The concern was due to the seeming quantum-safe properties
of lattice based cryptosystems. Since, ideal lattice-based schemes are
a small subset of these schemes there was a need to ensure that the
attack did not imply similar weaknesses in the more general schemes or
any other special cases of lattice-based schemes.