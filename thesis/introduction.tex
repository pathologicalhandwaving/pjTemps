\chapter{Introducing the Game}

Modern cryptographic schemes can be divided into two classes
asymmetric-key schemes and symmetric-key schemes. 
Symmetric-key schemes involve only one secret key between two communicating parties called a \emph{shared secret}. In contrast, asymmetric-key schemes contain two secret keys one for each party, and two public keys which are publicly distributed. (These schemes are discussed further in the text in greater detail).
\smallskip
Symmetric schemes are in general more efficient than asymetric
schemes but have the disadvantage that keys must be agreed upon by
parties who wish to communicate in advance. Specifically symmetric-key
schemes require that secret keys be distributed over secure channels.
These types of key distribution schemes are not compatible with most
modern methods of communication over networks, which are by and large
conducted over the internet, an insecure channel.
\smallskip
Public-key cryptography is an asymmetric-key class of systems which
allows parties which have never met, much less agreed upon secret keys
in advance, to communicate privately over insecure channels (we
continue this discussion in the background section of the text,
briefly). The primary concern is to make communication by public key
cryptographic schemes as efficient, secure, and easy to implement as
those of symmetric key systems.
\smallskip

In this context, use of symmetric encryption methods is often an implementation problem. 
That is, first the cryptosystem is constructed with some
thought in mind to its implementation and efficiency but with more
thought in mind to its security, particularly in terms of risk reduction of expected threats in the forseable future.

Since the primary reason cryptography exists is to keep
communications and information private the role of risk management is
weighted necessarily heavier towards ensuring the fulfillment of a
given set of security objectives.

An encryption scheme however is useless unless it can be implemented,
and a balance between efficiency and security must be carefully, and
continuously, maintained. The balance between security objectives and efficiency (usability) must be maintained throughout the development, implementation, and into daily use.

The balancing act between efficiency and security is illustrated in
the first half of this paper which gives an account of the events and
results following an unexpected announcement by British researchers in
the Communications-Electronics Security Group (CESG) (the information
assurance division of the Government Communications Headquarters
(GCHQ)) in their informal report Soliloquy: A Cautionary Tale\cite{cam20140}.
\newline
The introduction of lattice in cryptography brought in new,
enthusiastic researchers and propelled experienced researchers to the
forefront whose advice, often repeated, was misunderstood and in many instances ignored\cite{cra20151}.
\smallskip
It is worth explicitly stating that cryptography is inextricably bound and
defined by its relationship with security objectives and privacy. These
constraints not only benefit from the diversification schemes from
their original construction, they demand it.
\newline
However, as lattice cryptography increased in popularity the demand
for diversification was so often ignored by researchers in favor of
achieving efficiency it became increasingly difficult to find
constructions which had not been constructed over a particular number
field; more over research had focused not simply to this number field
it had narrowed to the point of most constructions being defined
specifically over the cyclotomic field of characteristic two\cite{cra20151}.
It is perhaps lucky that the discovery by the researchers at CESG was
not only disclosed publicly (the agency does not in general reveal much
less publish their findings), but that the defect was not found as a
consequence of an exploit, or even more at some point in the future
when cryptosystems having this type of construction may be widely
implemented and assumed secure.

\section{Introducing the Players}

Traditionally cryptography has been a two player game between
cryptographers and cryptanalysts. However, cryptanalysis has
diversified over time leaving cryptographers to play defense against an
increasingly diverse set of cryptanalysts on the offense.


