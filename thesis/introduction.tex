\chapter{Introducing the Game}

Modern cryptographic schemes can be viewed as two classes
asymmetric-key schemes and symmetric-key schemes.
Symmetric schemes are in general more efficient than asymetric
schemes but have the disadvantage that keys must be agreed upon by
parties who wish to communicate in advance. Specifically symmetric-key
schemes require that secret keys be distributed over secure channels.
These types of key distribution schemes are not compatible with most
modern methods of communication over networks, which are by and large
conducted over the internet, an insecure channel.

Public-key cryptography is an asymmetric-key class of systems which
allows parties which have never met, much less agreed upon secret keys
in advance, to communicate privately over insecure channels (we
continue this discussion in the background section of the text,
briefly). The primary concern is to make communication by public key
cryptographic schemes as efficient, secure, and easy to implement as
those of symmetric key systems.

The problem of mirroring the characteristics of symmetric key
schemes is a difficult problem and can be considered a benchmark for
implementation of public-key schemes.

In this context use of symmetric encryption methods is often a second
stage problem. That is, first the cryptosystem is created with some
thought in mind to its implementation and efficiency but with more
thought in mind to its security, particularly in the face of incoming
or expected threats in the near future.

Since the primary reason cryptography exists is to keep
communications and information private the role of risk management is
weighted necessarily heavier towards ensuring the fulfillment of a
given set of security objectives.

An encryption scheme however is useless unless it can be implemented
and a balance between efficiency and security must be carefully, and
continuously maintained throughout not only the development of the
scheme but during its implementation as well.

The balancing act between efficiency and security is illustrated in
the first half of this paper which gives an account of the events and
results following an unexpected announcement by British researchers in
the Communications-Electronics Security Group (CESG) (the information
assurance division of the Government Communications Headquarters
(GCHQ)) in their informal report Soliloquy: A Cautionary Tale.

The introduction of lattice in cryptography brought in new,
enthusiastic researchers and propelled experienced researchers to the
forefront whose advice, often repeated was misunderstood and in many
instances ignored.
It is worth explicitly stating that cryptography is inextricably bound and
defined by its relationship with security objectives and privacy. These
constraints not only benefit from the diversification schemes from
their original construction, they demand it.

However, as lattice cryptography increased in popularity the demand
for diversification was so often ignored by researchers in favor of
achieving efficiency it became increasingly difficult to find
constructions which had not been constructed over a particular number
field; more over research had focused not simply to this number field
it had narrowed to the point of most constructions being defined
specifically over the cyclotomic field of characteristic two.
It is perhaps lucky that the discovery by the researchers at CESG was
not only disclosed publicly (the agency does not in general reveal much
less publish their findings), but that the defect was not found as a
consequence of an exploit, or even more at some point in the future
when cryptosystems having this type of construction may be widely
implemented and assumed secure.

\subsection{Introducing the Players}

Traditionally cryptography has been a two player game between
cryptographers and cryptanalysts. However, cryptanalysis has
diversified over time leaving cryptographers to play defense against an
increasingly diverse set of cryptanalysts on the offense.

\subsubsection{Characterization of Cryptanalysts}

We divide cryptanalysts into four subclasses given as a variation of
the subclasses of cryptographic secrecy.
We make our division in reverse order in terms of hardness as
follows:

\begin{itemize}
\item{Practical Secrecy} In this subclass of cryptanalysts we have those
which most frequently occur. These cryptanalysts are in possesion of
finite computational resources but may use social, sidechannel, or
other methods to obtain a break on a system which does not break the
mathematics of the scheme itself, may not in the future be reproduced,
but are sucessful in their attempt none the less.

\item{Computational Secrecy} This subclass of cryptanalysts are those who
we can assume have at their disposal more resources and are capable of
more large scale attacks than those in the previous subclass. They are
still in possession of all the methods of the practical cryptanalysts
but we make the assumption that they are not working alone but as a
well organized unit such as those of a large organization. These
attacks are becoming more and more frequent the organization itself may
be known to the defense but for various reasons ( including but not
limited to political) a direct confrontation must be avoided.

\item{Unconditional Secrecy} In this subclass we find those cryptanalysts
who directly attack the mathematics of a given cryptosystem.
Specifically we have the cryptanalysts who play against most modern
systems based on the discrete logarithm problem, the integer
factorization problem, and other number theoretic schemes. We limit
this group to problems which are played over systems whose proofs of
security are given over schemes an average-case hardness.

\item{Perfect Secrecy} This is the subclass of cryptanalysts we will be
concerned with for the remainder of this paper. These players are
concerned with breaking the mathematics of schemes based on intractible
problems with a provable security based on worst-case hardness
assumptions. Specifically, we concern ourselves with the concept of
quantum-safe cryptographic schemes over lattices.
\end{itemize}

\subsection{The Soliloquy Problem}

In 2010 at the International Public Key Cryptography conference Smart
and Veracauteren introduced a cryptographic scheme [Sma2010] based on
principal ideal lattices with relatively small key sizes and promising
to enable FHE.

In late 2014 GCHQ announced that they had developed an equivalent
cryptosystem to [Sma2010] in 2007 named Soliloquy, but from 2010 to
2013 they had constructed a successful attack against the system and
decided to cease development.

Furthermore, they claimed (without proof) that the assumed hard
problem of finding the short generator of a principal ideal lattice was
both easy and efficient to solve. This conflicted with the [Sma2010]
claim that the problem of finding a short generator was hard. In
addition the solvability claim GCHQ was not only given for quantum
computers but also for ordinary modern computer system with a practical
amounts of resources.

The announcement by GCHQ created quite a stir in the cryptographic
community. The concern was due to the seeming quantum-safe properties
of lattice based cryptosystems. Since, ideal lattice-based schemes are
a small subset of these schemes there was a need to ensure that the
attack did not imply similar weaknesses in the more general schemes or
any other special cases of lattice-based schemes.

\subsection{Overview of Paper Contents}

In the first half of this paper we give an overview of fully
homomorphic encryption (FHE) as well as describe the need for such
systems.
We give a summary of the history of FHE schemes and an overview of
the first sucessful construction given by Gentrys Stanford PhD thesis
[Gen2009] as well as the construction from [Sma2010].
We contrast these two schemes with an overview of the more general
construction of Peikerts RLWE and discuss the LWE problem, and the
concept of the mathematical basis of hardness assumptions.
We then give an overview of the quantum and general solutions proved
since the GCHQ announcement and summarize the results found by
researchers in the process of verification of these claims.
In the second half of this paper we discuss the problem of verifying
if RLWE schemes come with a connected expander graph and its
implication that FHE is circuit private.

